\chapter{Conclusion and Discussion}
\label{cha:conclusion_and_discussion}
The problem of connecting external data sources to the blockchain can be solved in many ways. There are various types of Oracles available, and they are all suited for different use cases. For web applications that need to query basic APIs that can be reached via HTTP requests, the Oraclize service is a suitable solution.

Other services that support more complex scenarios, such as ChainLink with its external modules may be used in more complex applications, and applications that require authentication to the data source.

For most use cases, a solution already exists on the market. In this way, applications on the blockchain, or so called "Dapps" can become a lot more useful. As these connectors grow in functionality, adoption and variety, one can foresee that those kinds of apps will also see an increase in popularity. There are already a great number of ideas that have been implemented in a basic way out there, but they are not well known or adopted yet. Examples are \emph{DTube}, a decentralized video platform, \emph{Steemit}\cite{steemitwhitepaper}, a decentralized social media platform, \emph{IPFS}, the InterPlanetary File System\cite{ipfswhitepaper}, which is used by D.tube, and many more. 

On the contrary side, the disadvantages and shortcomings of blockchain applications are lower throughput than centralized services, and in many cases the user friendliness is not at a satisfactory level yet. As developers become more familiar with these technologies, and users get accustomed to working with those kind of apps, these shortcomings can be overcome. The technology is innovative and disruptive, and, by revolutionizing the way trust works on the internet, clearly has a bright future.