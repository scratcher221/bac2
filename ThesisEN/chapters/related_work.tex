\chapter{Related Work}
\label{cha:related_work}
\section{Managing Data on the Blockchain}
There are numerous existing implementations of decentralized apps, some of which we will briefly explore in this section. One example is \emph{MedRec} a system for storing and managing medical data on the blockchain \cite{azaria2016medrec}.

Managing a web application through an administrator backend access is often associated with a lot of time and effort, since the administrator or a team of managers constantly have to monitor the system, and push changes manually as needed. An example of how management can be made easier and more efficient is demonstrated in the system \emph{MedRec}.

Specifically, in this system permission management is done completely automatic by contracts designed for this purpose. A patient can decide what parts of their medical data can be accessed by which parties. It is not necessary to rely on a central server, managed by a trusted authority, to control who can view the data and to which extent. Instead, this information is stored on the distributed ledger, and thus enforced by all the participating nodes. No additional administration is needed, the two parties consisting of patient and care provider can interact directly with each other. Furthermore, the parties don't have to interact with the blockchain directly, as this would be not very user friendly, but an interface is provided which translates the user input into the appropriate API calls that form the connection to the contracts. Querying of data is also done through the blockchain, i.e. the SQL strings for retrieving the data are stored there, and sent to an application for further processing if triggered by an authorized party.

This reduction of management overhead and increase in efficiency can be extended to the application accompanying this work, the cryptocurrency sportsbetting app. The administrators only have to provide information about the available games users can bet on. Other tasks, like managing user funds, access control to the total funds in the contract, and who can withdraw under which circumstances are implemented in the contracts.

\section{Security Tools for Smart Contracts}
Programming in the realm of smart contracts on decentralized networks always involves dealing with money in the code, as we have already demonstrated in chapter \ref{sec:design_choices}. Because of this, security has a very high priority, to ensure that users can trust in the system and funds are not at risk of being stolen by hackers. Another reason for the need of improved security measures is the fact that a lot of smart contract platforms, including Ethereum, are completely open to the public. Anyone with internet access can call contracts if they have the address, so the worst-case scenario always needs to be considered when developing apps for those platforms. Especially if a \emph{Dapp} gains popularity, it has to be assumed that hackers are interested in exploiting the system for personal profit.

Another issue we need to focus on, is the fact that network participants (miners), are the ones who decide which transactions are accepted, how they should be ordered and they also set the block timestamp, which can be a source of manipulation.

There are a variety of security scanning tools available, mostly for code written for the Ethereum Network. We will explore one of these tools, \emph{Oyente}, in more detail in this section.

\section*{The Security Scanning Tool Oyente}

\emph{Oyente} is a symbolic execution tool developed by Luu et al. and described in their work "Making Smart Contracts Smarter"\cite{luu2016making}. It is capable of analyzing EVM bytecode directly, without needing access to the high level representation in Solidity or Serpent. This is important, because often access to the high level code is not available, as the Ethereum blockchain only stores the EVM bytecode and in many cases developers don't provide access to public repositories to review the source code. However, the tool is also capable of analyzing source code directly, so one can specify a Solidity source code file as input and the tool will scan this contract for critical bugs. The code for \emph{Oyente} is open source and can be found on GitHub \cite{oyentegithub}.

The easiest way to start using Oyente is to make use of the pre-fabricated docker image that is available under \texttt{luongnguyen/oyente}. This image contains all the necessary dependencies for the application, which makes the installation process very easy. We can download the image by issuing the command \texttt{docker pull luongnguyen/oyente} on the terminal, and then creating a new container from this image by executing \texttt{docker run -i -t luongnguyen/oyente}, or better starting the container in privileged mode with \texttt{docker run -{}-privileged -i -t luongnguyen/oyente}, to have access to all devices on the host. It should be noted that root access is required to interact with the docker daemon.

Inside this container, the Python program \texttt{oyente.py} is available in the directory \texttt{/oyente/oyente/}. We can evaluate a contract by changing into this directory and then running \texttt{python oyente.py -s Contract.sol}, for Solidity source code. To evaluate a contract that is only available in EVM bytecode, the flag "-b" needs to be appended to the command. For example, \texttt{python oyente.py -s BytecodeContract -b} evaluates a file which contains EVM bytecode.

Since Oyente runs in a docker container that is created from the image from scratch every time, in order to make changes in this container permanent, the modified container needs to be saved explicitly. The starting container has example Solidity contracts for analyzing, but to run the tool on user defined contracts, those need to be copied into the container first. This is done by obtaining the container ID via \texttt{sudo docker ps} and then executing \texttt{sudo docker cp [source\textunderscore path] [container\textunderscore id]:[destination\textunderscore path]}.

The version of the Solidity compiler \texttt{solc} which is found in the container created from the image is outdated. Because of this, it is recommended to update all software packages in the container by running first \texttt{apt-get update}, and then \texttt{apt-get upgrade}. After making all these changes, the modified container can be saved by executing \texttt{sudo docker commit [container\textunderscore id] luongnguyen/oyente:version2}.

\section*{Analyzing the Contracts in our Project}

To ensure the security of the smart contracts powering our cryptocurrency sportsbetting application, we make use of the aforementioned scanning tool. Running \emph{Oyente} on the \emph{BetManager} contract, which inherits from all other contracts, produces a satisfactory, albeit not perfect output. There were no extremely critical bugs, such as \emph{Parity Multisig Bug, Callstack Depth Attack Vulnerability, Transaction-Ordering Dependence, Timestamp Dependency} or \emph{Re-Entrancy Vulnerability}. However the scanning tool produced a positive result concerning \emph{Integer Underflow} and \emph{Integer Overflow} for certain contracts.
\begin{table}[ht]
	\centering
	\begin{tabular}{ | l | l | r |}
		\hline
		\multicolumn{3}{ | c | }{\textbf{Discovered Vulnerabilities}} \\ \hline
		\emph{Affected contract} & \emph{Integer Underflow} & \emph{Integer Overflow} \\ \hline
		Bet Manager & True & True \\ \hline
		Betting Factory & False & True \\ \hline
		Game Manager & True & True \\ \hline
		Ownable & False & False \\
		\hline
	\end{tabular}
	\caption{\label{tab:contract-bugs}Overview of discovered vulnerabilities.}
\end{table}
\\
Upon closer inspection, some of these overflows could be fixed by using the \texttt{SafeMath} library for mathematical operations. However, some of the positive results were simply false positives that could not be reproduced.
