 \chapter{The Problem of Trust and Centralization}
 \label{cha:trust_and_centralization}
 \section{Why Centralized Control is Undesirable in this Project}
 The reason for building a decentralized application is mainly that users don't have to trust a central authority with their money and their data. Another reason is that users can rely on the integrity of the data provided by the distributed system. In the fully decentralized peer-to-peer model, there is no central point that can be attacked or tampered with. So the end user can be sure that the data they received from one of the nodes of the network is authentic, or at least they can verify this with digital signatures.
 
 However, if there is a central instance that provides the data, as is the case in the current state of the project, users can never be sure that this central source has not been compromised. As of now, the power lies in the hands of the administrators who control what games users can bet on, and also the results of these games. While security precautions have been taken to ensure that not \emph{anyone} can alter game results or manipulate bets, since only the creators of the \texttt{BetManager} contract on the blockchain can call the functions responsible for this, there is still the possibility of the administrators being bribed, or their private keys being stolen. Furthermore, the administrators could take advantage of their power and use their control for personal financial gain, by providing game results they have bet on, etc.
 
 For all these reasons mentioned above, in its current state, the project ultimately defeats the purpose of being provided on a decentralized platform because it still sources crucial data from a central hand. The whole point of blockchain systems is to not have this central point. Systems built based on the traditional model with a central instance have many advantages like better speed, latency and throughput compared to those novel platforms. The factor of having no single controlling instance is essential and must be preserved in order to not defeat the purpose of our project. The target user, which is someone who rejects centralization in favor of decentralized, distributed systems, can not be satisfied with the system as it is. Countermeasures need to be taken and solutions provided to mitigate these issues. This is where so called \emph{Blockchain Oracles} come in.
 \section{The Concept of Blockchain Oracles}
 In order to receive data from the world outside of the Ethereum network, we need a way for smart contracts to be able to interact with regular web servers, for example through APIs that can be queried over regular HTTP or HTTPS. A blockchain oracle is basically just a connector that makes this possible. Usually, the oracle itself is also a contract deployed on the blockchain, that can be accessed at its contract address, and queried through specific methods.
