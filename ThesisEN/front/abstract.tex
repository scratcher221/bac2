\chapter{Abstract}


Applications on the blockchain are becoming increasingly popular, since the introduction of the Ethereum Network. One of the main issues with those applications, also called "Dapps", is the gathering of data from the outside world. Methods governing the functionality of the contracts on the network still have to be called by humans who supply the necessary input data. This is obviously a great weakness in a decentralized system, since the owner of a contract can exploit this by providing manipulated data that suits his or her interest.

For instance, contracts that carry out a certain action based on the occurrence of a certain event, need a reliable way to determine if that event did in fact happen or not. Relying on an administrator or system supervisor to decide whether or not it happened completely defeats the purpose of a decentralized system. One might as well fall back on traditional structures of computer networks with a central server in that case, because in terms of efficiency, scalability and speed those are superior to decentralized solutions.

For those reasons, a reliable way of gathering data without a central actor is needed. This can be accomplished with so-called "Blockchain Oracles", which connect applications on the blockchain to various data feeds available on the internet. There are different providers for oracles. Usually they work by providing an API which can be integrated into the smart contracts. The contracts can then be programmed to poll different kinds of web services such as weather information providers, news outlets, etc. by communicating through this API. The goal is to have a method for getting this data on demand, by using one or more oracle services that are available. Different methods for obtaining this goal are explored in this work. 
